\documentclass[12pt,a4paper]{article}
\usepackage[utf8]{inputenc}
\usepackage[russian]{babel}
\usepackage{amsmath}
\usepackage{amsfonts}
\usepackage{amssymb}
\usepackage[left=2cm,right=2cm,top=2cm,bottom=2cm]{geometry}
\usepackage{graphicx}
\begin{document}


\begin{enumerate}
\item  По совету Лисы Волк опустил в прорубь хвост и поймал 100 чудо-рыб. Веса рыбин независимы и имеют распределение Вейбулла, $f(x)=2\exp(-x^2/a^2)\cdot x/a^2$ при $x\geq 0$. Известно, что $\sum x_i^2=120$.
\begin{enumerate}
\item Найдите ML оценку параметра $a$
\item Постройте 95\% доверительный интервал для $a$
\item С помощью LR, LM и W теста проверьте гипотезу о том, что $a=1$.
\end{enumerate}

\item Как известно, Фрекен-Бок пьет коньяк по утрам и иногда видит привидения. За 110 дней имеются следующие статистические данные


\begin{tabular}{c|ccc}
Рюмок & 1 & 2 & 3 \\ 
\hline 
Дней с привидениями & 10 & 25 & 20 \\ 
Дней без привидений & 20 &  25 & 10 \\ 
\end{tabular}

Вероятность увидеть привидение зависит от того, сколько рюмок коньяка было выпито утром, а именно, $p=\exp(a+bx)/(1+ \exp(a+bx))$, где $x$ --- количество рюмок, а $a$ и $b$ --- неизвестные параметры.
\begin{enumerate}
\item Найдите\footnote{Здесь потребуется максимизировать функцию в R. Если этот пункт не получился, то в последующих пунктах можно считать, что $\hat{a}=-1.5$, а $\hat{b}=0.5$. Это сильно округленные значения коэффициентов.} ML оценки неизвестных параметров $a$ и $b$. 
\item Постройте 95\%-ые доверительные интервалы для $a$ и $b$
\item С помощью LR, LM и W теста проверьте гипотезу о том, что $b=0$.
\item С помощью LR, LM и W теста проверьте гипотезу о том, что $a=0$ и одновременно $b=0$. 
\end{enumerate}

\end{enumerate}


\vspace{20pt}

Всем участникам переписывания правдоподобной контрольной счастья! Много!

\vspace{20pt}

Сегодня, 20 марта, \textbf{Международный День счастья}.


\end{document}