\documentclass[12pt,a4paper]{article}

\usepackage[top=3cm, left=2cm, right=2cm]{geometry} % размер текста на странице

\usepackage{tikz} % картинки в tikz
\usepackage{microtype} % свешивание пунктуации

\usepackage{array} % для столбцов фиксированной ширины

\usepackage{indentfirst} % отступ в первом параграфе

\usepackage{sectsty} % для центрирования названий частей
\allsectionsfont{\centering}

\usepackage{amsmath} % куча стандартных математических плюшек

\usepackage{multicol} % текст в несколько колонок

\usepackage{lastpage} % чтобы узнать номер последней страницы

\usepackage{enumitem} % дополнительные плюшки для списков
%  например \begin{enumerate}[resume] позволяет продолжить нумерацию в новом списке

\usepackage{amsmath}
\usepackage{amssymb}


\usepackage{fontspec}
\usepackage{polyglossia}

\setmainlanguage{russian}
\setotherlanguages{english}

% download "Linux Libertine" fonts:
% http://www.linuxlibertine.org/index.php?id=91&L=1
\setmainfont{Linux Libertine O} % or Helvetica, Arial, Cambria
% why do we need \newfontfamily:
% http://tex.stackexchange.com/questions/91507/
\newfontfamily{\cyrillicfonttt}{Linux Libertine O}

\AddEnumerateCounter{\asbuk}{\russian@alph}{щ} % для списков с русскими буквами




\AddEnumerateCounter{\asbuk}{\russian@alph}{щ}
%\renewcommand{\theenumi}{\asbuk{enumi}}
\renewcommand{\theenumii}{\asbuk{enumii}}


% \usepackage[left=1cm,right=1cm,top=1cm,bottom=1cm]{geometry}

\usepackage{fancyhdr} % весёлые колонтитулы
\pagestyle{fancy}
\lhead{Эконометрика, комиссия}
\chead{}
\rhead{10.10.2016}
\lfoot{}
\cfoot{}
\rfoot{\thepage/\pageref{LastPage}}
\renewcommand{\headrulewidth}{0.4pt}
\renewcommand{\footrulewidth}{0.4pt}

\DeclareMathOperator{\tr}{tr}
\DeclareMathOperator{\E}{\mathbb{E}}
\let\P\relax
\DeclareMathOperator{\P}{\mathbb{P}}
\DeclareMathOperator{\Var}{\mathbb{V}ar}
\DeclareMathOperator{\hVar}{\widehat{\mathbb{V}ar}}
\newcommand{\hb}{\hat\beta}
\newcommand{\hs}{\hat\sigma}

\DeclareMathOperator{\Cov}{\mathbb{C}ov}

\begin{document}



\begin{enumerate}
\item Пусть $y_i = \beta_1 + \beta_2 x_i + \varepsilon_i$ и $i = 1, \dots, 5$ — классическая регрессионная модель. Также имеются следующие данные: $\sum_{i=1}^5 y_i^2 = 100, \sum_{i=1}^5 x_i^2 = 100, \sum_{i=1}^5 x_iy_i = 12, \sum_{i=1}^5 y_i = 15, \sum_{i=1}^5 x_i = 3.$

\begin{enumerate}
\item Найдите $\hat{\beta_1}$ и $\hat{\beta_2}$
\item Найдите $TSS$, $ESS$, $RSS$ и $R^2$
\item Предполагая нормальность случайной составляющей проверьте на уровне значимости 5\% гипотезу $H_0$: $\beta_2 = 0$.
\end{enumerate}


\item Домохозяйка Глаша очень любит читать романы Л.Н.~Толстого и смотреть сериалы. Её сын Петя учится на третьем курсе ВШЭ.  Последние 30 дней он записывал, сколько Глаша прочитала страниц «Анны Карениной», $pages_t$, и посмотрела серий «Доктора Хауса», $series_t$. На основании этих наблюдений при помощи МНК Петя оценил следующую модель:

\[
\widehat{pages}_t=200-3series_t
\]

Оценка ковариационной матрицы коэффициентов,
$\hVar(\hb) = \begin{pmatrix}
11 & 1 \\
1 & 2 \\
\end{pmatrix}$

Оценка дисперсии ошибок равна $\hs^2=323$.

Завтра Глаша собирается посмотреть 10 серий «Доктора Хауса».

\begin{enumerate}
\item Постройте точечный прогноз количества прочитанных Глашей страниц романа
\item Постройте 95\%-ый доверительный интервал для $\E(pages_t |series_t=10)$, ожидаемого количества прочитанных страниц
\item	Постройте 95\%-ый предиктивный интервал для фактического количества прочитанных страниц
\end{enumerate}


\item В модели $y_i = \beta_1 + \beta_2 x_i + u_i$ выполнены все предпосылки теоремы Гаусса-Маркова кроме одной, а именно, $\Var(u_i) = i^2 \sigma^2$.
\begin{enumerate}
  \item Какими свойствами будут обладать обычные МНК-оценки в этой модели?
  \item Подробно опишите, какой метод разумно применить в этой задаче помимо обычного МНК. Какими свойствами будут обладать оценки этого метода? Если метод требует построения регрессий, то подробно опишите, как выглядит вектор зависимых переменных и матрица регрессоров в каждой оцениваемой регрессии.
\end{enumerate}



\item Сформулируйте логит-модель. Опишите метод, которым оцениваются коэффициенты в логит-модели. Явно выпишите целевую функцию. Опишите как оцениваются стандартные ошибки коэффициентов в логит-модели.


\end{enumerate}

\end{document}
