\documentclass[pdftex,12pt,a4paper]{article}\usepackage[]{graphicx}\usepackage[]{color}
%% maxwidth is the original width if it is less than linewidth
%% otherwise use linewidth (to make sure the graphics do not exceed the margin)
\makeatletter
\def\maxwidth{ %
  \ifdim\Gin@nat@width>\linewidth
    \linewidth
  \else
    \Gin@nat@width
  \fi
}
\makeatother

\definecolor{fgcolor}{rgb}{0.345, 0.345, 0.345}
\newcommand{\hlnum}[1]{\textcolor[rgb]{0.686,0.059,0.569}{#1}}%
\newcommand{\hlstr}[1]{\textcolor[rgb]{0.192,0.494,0.8}{#1}}%
\newcommand{\hlcom}[1]{\textcolor[rgb]{0.678,0.584,0.686}{\textit{#1}}}%
\newcommand{\hlopt}[1]{\textcolor[rgb]{0,0,0}{#1}}%
\newcommand{\hlstd}[1]{\textcolor[rgb]{0.345,0.345,0.345}{#1}}%
\newcommand{\hlkwa}[1]{\textcolor[rgb]{0.161,0.373,0.58}{\textbf{#1}}}%
\newcommand{\hlkwb}[1]{\textcolor[rgb]{0.69,0.353,0.396}{#1}}%
\newcommand{\hlkwc}[1]{\textcolor[rgb]{0.333,0.667,0.333}{#1}}%
\newcommand{\hlkwd}[1]{\textcolor[rgb]{0.737,0.353,0.396}{\textbf{#1}}}%

\usepackage{framed}
\makeatletter
\newenvironment{kframe}{%
 \def\at@end@of@kframe{}%
 \ifinner\ifhmode%
  \def\at@end@of@kframe{\end{minipage}}%
  \begin{minipage}{\columnwidth}%
 \fi\fi%
 \def\FrameCommand##1{\hskip\@totalleftmargin \hskip-\fboxsep
 \colorbox{shadecolor}{##1}\hskip-\fboxsep
     % There is no \\@totalrightmargin, so:
     \hskip-\linewidth \hskip-\@totalleftmargin \hskip\columnwidth}%
 \MakeFramed {\advance\hsize-\width
   \@totalleftmargin\z@ \linewidth\hsize
   \@setminipage}}%
 {\par\unskip\endMakeFramed%
 \at@end@of@kframe}
\makeatother

\definecolor{shadecolor}{rgb}{.97, .97, .97}
\definecolor{messagecolor}{rgb}{0, 0, 0}
\definecolor{warningcolor}{rgb}{1, 0, 1}
\definecolor{errorcolor}{rgb}{1, 0, 0}
\newenvironment{knitrout}{}{} % an empty environment to be redefined in TeX

\usepackage{alltt}


% размер листа бумаги
\usepackage[paper=a4paper,top=13.5mm, bottom=13.5mm,left=16.5mm,right=13.5mm,includefoot]{geometry}
\usepackage{makeidx} % создание индекса
\usepackage{cmap} % поиск русских букв в pdf 
% \usepackage[pdftex]{graphicx} % omit pdftex option if not using pdflatex
\usepackage[colorlinks,hyperindex,unicode]{hyperref}

\usepackage[utf8]{inputenc}
\usepackage[T2A]{fontenc} 
\usepackage[russian]{babel}

\usepackage{amssymb}
\usepackage{amsmath}
\usepackage{amsthm}
\usepackage{epsfig}
\usepackage{bm}
\usepackage{color}

\usepackage{textcomp}  % Чтобы в формулах можно было русские буквы писать через \text{}
\usepackage{multicol}
\usepackage{enumitem} % дополнительные плюшки для списков
%  например \begin{enumerate}[resume] позволяет продолжить нумерацию в новом списке

\def \b{\beta}
\def \hb{\hat{\beta}}
\def \hs{\hat{s}}
\def \hy{\hat{y}}
\def \hY{\hat{Y}}
\def \he{\hat{\varepsilon}}
\def \v1{\vec{1}}
\def \e{\varepsilon}
\DeclareMathOperator{\Cov}{Cov}
\DeclareMathOperator{\Var}{Var}
\def \hVar{\widehat{\Var}}
\def \hCorr{\widehat{\Corr}}
\def \hCov{\widehat{\Cov}}
\def \P{\mathbb{P}}
\def \E{\mathbb{E}}







\title{Метрика. Полуфинал. 26 декабря 2014.}
\IfFileExists{upquote.sty}{\usepackage{upquote}}{}
\begin{document}
%\maketitle
\parindent=0 pt % no indent

\section*{Метрика. Ликвидация безграмотности --- 2014}

В этот день, 26 декабря 1919 года, совнарком РСФСР принял декрет <<О ликвидации безграмотности в РСФСР>>. Всем желаю отметить этот день написанием грамотного зачета по эконометрике! Удачи!

\begin{enumerate}
\item Регрессионная модель  задана в матричном виде при помощи уравнения $y=X\beta+\varepsilon$, где $\beta=(\beta_1,\beta_2,\beta_3)'$.
Известно, что $\E(\varepsilon)=0$  и  $\Var(\varepsilon)=\sigma^2\cdot I$.
Известно также, что 

$y=\left(
\begin{array}{c} 
1\\ 
2\\ 
3\\ 
4\\ 
5
\end{array}\right)$, 
$X=\left(\begin{array}{ccc}
1 & 0 & 0 \\
1 & 0 & 0 \\
1 & 0 & 0 \\
1 & 1 & 0 \\
1 & 1 & 1 
\end{array}\right)$.


Для удобства расчетов приведены матрицы 


$X'X=\left(
\begin{array}{ccc} 
5 & 2 & 1\\ 
2 & 2 & 1\\ 
1 & 1 & 1 
\end{array}\right)$ и $(X'X)^{-1}=\frac{1}{3}\left(
\begin{array}{ccc} 
1 & -1 & 0 \\
-1 & 4 & -3 \\
0 & -3 & 6
\end{array}\right)$.

\begin{enumerate}
\item Найдите вектор МНК-оценок коэффициентов $\hb$.
\item Найдите коэффициент детерминации $R^2$
\item Предполагая нормальное распределение вектора $\varepsilon$, проверьте гипотезу $H_0$: $\b_2=0$ против альтернативной $H_a$: $\b_2\neq 0$
\end{enumerate}

\item Для линейной регрессии $y_i = \beta_1 + \beta_2 x_i + \beta_3 z_i + \e_i$ была выполнена сортировка наблюдений по возрастанию переменной $x$. Исходная модель оценивалась по разным частям выборки:

\begin{tabular}{c|cccc}
Выборка & $\hb_1$ & $\hb_2$ & $\hb_3$ & $RSS$ \\

\hline 
$i=1,\ldots, 50$ & $0.93$ & $2.02$ & $3.38$ & $145.85$ \\ 
$i=1,\ldots, 21$ & $1.12$ & $2.01$ & $3.32$ & $19.88$ \\ 
$i=22,\ldots, 29$ & $0.29$ & $2.07$ & $2.24$ & $1.94$ \\ 
$i=30,\ldots, 50$ & $0.87$ & $1.84$ & $3.66$ & $117.46$ \\ 
\end{tabular} 

Известно, что ошибки в модели являются независимыми нормальными случайными величинами с нулевым математическим ожиданием. 

\begin{enumerate}
\item Предполагая гомоскедастичность остатков на уровне значимости 5\% проверьте гипотезу, что исследуемая зависимость одинакова на всех трёх частях всей выборки.
\item Протестируйте ошибки на гетероскедастичность на уровне значимости 5\%.
\item Какой тест можно на гетероскедастичность можно было бы использовать, если бы не было уверенности в нормальности остатков? Опишите пошагово процедуру этого теста.
\end{enumerate}

\newpage

\item По 2040 наблюдениям оценена модель зависимости стоимости квартиры в Москве (в 1000\$) от общего метража и метража жилой площади.
% latex table generated in R 3.1.1 by xtable 1.7-4 package
% Fri Dec 26 08:52:31 2014
\begin{table}[ht]
\centering
\begin{tabular}{rrrrr}
  \hline
 & Estimate & Std. Error & t value & Pr($>$$|$t$|$) \\ 
  \hline
Константа & -88.81 & 4.37 & -20.34 & 0.00 \\ 
  Общая площадь & 1.70 & 0.10 & 17.78 & 0.00 \\ 
  Жилая площадь & 1.99 & 0.18 & 10.89 & 0.00 \\ 
   \hline
\end{tabular}
\end{table}


Оценка ковариационной матрицы $\widehat{Var}(\hat{\beta})$ имеет вид
% latex table generated in R 3.1.1 by xtable 1.7-4 package
% Fri Dec 26 08:52:31 2014
\begin{table}[ht]
\centering
\begin{tabular}{rrrr}
  \hline
 & (Intercept) & totsp & livesp \\ 
  \hline
(Intercept) & 19.07 & 0.03 & -0.45 \\ 
  totsp & 0.03 & 0.01 & -0.02 \\ 
  livesp & -0.45 & -0.02 & 0.03 \\ 
   \hline
\end{tabular}
\end{table}


Оценка стандартной ошибки случайной составляющей, $\hat{\sigma}=33.0253$.

\begin{enumerate}
\item Можно ли интерпретировать коэффициент при переменной $totsp$ как стоимость одного метра нежилой площади?
\item Проверьте гипотезу о том, что коэффициенты при регрессорах $totsp$ и $livesp$ равны.
\item Постройте 95\%-ый доверительный интервал для ожидаемой стоимости квартиры с жилой площадью $30$ м$^2$ и общей площадью $60$ м$^2$.
\item Постройте 95\%-ый прогнозный интервал для фактической стоимости квартиры с жилой площадью $30$ м$^2$ и общей площадью $60$ м$^2$.
\end{enumerate}

\item Аккуратно сформулируйте теорему Гаусса-Маркова 
\begin{enumerate}
\item для нестохастических регрессоров
\item для стохастических регрессоров в предположении, что наблюдения являются случайной выборкой
\end{enumerate}




\end{enumerate}

\end{document}
