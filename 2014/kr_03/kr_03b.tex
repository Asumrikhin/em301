\documentclass[12pt,a4paper]{article}
\usepackage[utf8]{inputenc}
\usepackage[russian]{babel}

\usepackage{amsmath}
\usepackage{amsfonts}
\usepackage{amssymb}
\usepackage[left=2cm,right=2cm,top=2cm,bottom=2cm]{geometry}

\DeclareMathOperator{\Var}{Var}
\DeclareMathOperator{\Cov}{Cov}
\DeclareMathOperator{\plim}{plim}

\newcommand{\hy}{\hat{y}}
\newcommand{\hb}{\hat{\beta}}
\newcommand{\e}{\varepsilon}
\newcommand{\E}{\mathbb{E}}
\renewcommand{\P}{\mathbb{P}}
\begin{document}

Праздник по эконометрике номер 3 



Во всех задачах, если явно не сказано обратное, предполагается, что выполнены стандартные предпосылки классической линейной регрессионной модели. 

\begin{enumerate}

\item  Эконометресса Ефросинья исследует зависимость удоев от возраста и породы коровы. Она оценила модель 
\[
\hy_i = \hb_1 + \hb_2 age_i +\hb_3 d_{1i} + \hb_4 d_{2i} 
\]
Эконометресса Глафира исследует ту же зависимость:
\[
\hy_i = \hb'_1 + \hb'_2 age_i +\hb'_3 d'_{1i} + \hb'_4 d'_{2i} 
\]
но вводит дамми-переменные вводит по-другому:

\begin{tabular}{c|cccc}
Порода коровы & $d_1$ & $d_2$ & $d'_1$ & $d'_2$  \\ 
\hline 
Холмогорская & 0 & 0 & 1 & 1 \\ 
Тагильская & 1 & 0 & 0 & 1  \\ 
Ярославская & 0 & 1 & 1 & 0  \\ 
\hline 
\end{tabular} 

\begin{enumerate}
\item Выразите оценки коэффициентов Глафиры через оценки коэффициентов Ефросиньи.
\item Какую регрессию помимо исходной надо оценить Глафире, чтобы проверить гипотезу, что удои у Холмогорской и Тагильской пород совпадают?
\item Допустим, что в исходной модели $R^2=0.9$, а во второй, упомянутой в (b), $R^2=0.8$. Модели оцениваются по 70 наблюдениям. Проверьте гипотезу пункта (b) на уровне значимости 5\%.
\end{enumerate}



\item По 24 наблюдениям, соответствующим 24 представительствам национальной компании по торговле недвижимостью, была оценена регрессия объёма годовых продаж (Sales, млн. долл.) на число агентов в представительстве (Agents) и объём затрат на рекламу (AdvCosts, тыс. долл.).
Ниже приведены результаты оценивания:
\[
\widehat{Sales}_i = -7.7 + 0.3 AdvCosts_i + 0.8Agents_i,  \; RSS=42
\]
\[
(X'X)^{-1}=\begin{pmatrix}
3.2 & 0.003 & 0.3 \\
0.003& 0.005 & 0.002 \\
0.3 & 0.002 & 0.4
\end{pmatrix}
\]
\begin{enumerate}
\item Найдите оценку дисперсии случайной составляющей.
\item  Найдите оценку дисперсии коэффициента перед AdvCosts.
\item  Проверьте значимость коэффициента при затратах на рекламу на уровне значимости 1\%
\item Компания планирует открыть новое представительство с 20 агентами и годовыми затратами на рекламу в миллион долларов. Постройте 95\%-ый предиктивный интервал для объёма годовых продаж нового представительства.
\end{enumerate}

\item Имеется $100$ наблюдений. Исследователь Вениамин предполагает, что дисперсия случайной ошибки непостоянна и подчиняется закону $\Var(\e_t)=t\sigma^2$. Вениамин оценивает модель $y_t=\beta_1 + \beta_2 x_t +\varepsilon_t$ с помощью МНК.
\begin{enumerate}
\item Найдите истинную дисперсию МНК оценки коэффициента $\beta_2$
\item Предложите более эффективную оценку $\hat{\beta_2}^{alt}$
\item Подробно опишите любой способ, который позволяет протестировать гипотезу о гомоскедастичности против предположения Вениамина о дисперсии.
\end{enumerate}
\newpage

\item В модели парной регрессии $y_i=\beta_1+\beta_2 x_i +\e_i$ ошибки $\e_i$ независимы и имеют пуассоновское распределение с параметром $\lambda$.
\begin{enumerate}
\item Предложите способ несмещенно оценить $\lambda$.
\item Являются ли МНК-оценки $\hb_1$ и $\hb_2$ несмещенными? Если оценки являются смещенными, то предложите несмещенные оценки
\end{enumerate}


\item Эконометресса Эвридика хочет оценить модель $y_i=\beta_1 + \beta_2 x_i +\beta_3 z_i + \e_i$. К сожалению, она измеряет зависимую переменную с ошибкой. Т.е. вместо $y_i$ она знает значение $y_i^*=y_i+u_i$ и использует его в качестве зависимой переменной при оценке регрессии. Ошибки измерения $u_i$ некоррелированы между собой и с $\e_i$, имеют нулевое математическое ожидание и постоянную дисперсию $\sigma^2_u$.
\begin{enumerate}
\item Будут ли оценки Эвридики несмещенными?
\item Могут ли дисперсии оценок Эвридики быть ниже чем дисперсии МНК оценок при использовании настоящего $y_i$?
\item Могут ли оценки дисперсий оценок Эвридики быть ниже чем оценок дисперсий МНК оценок при использовании настоящего $y_i$?
\end{enumerate}


\end{enumerate}



\end{document}